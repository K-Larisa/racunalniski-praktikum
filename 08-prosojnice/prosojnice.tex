\documentclass{beamer}
% Naloga 1.3.1: Za dokument uporabite razred `beamer'.
% Ne dodajajte nastavitve za velikost pisave, kot je bila v datoteki `5-prosojnice.tex`.
\usepackage[T1]{fontenc}
\usepackage[utf8]{inputenc}
\usepackage[slovene]{babel}


% Naloga 1.3.2: vključite paket `predavanja'.
\usepackage{pgfpages}


%naslednje tri zadeve so vzete iz datoteke 5-prosojnice.tex
% Malo manj dolgočasna pisava
\usepackage{palatino}
\usefonttheme{serif}
% Izklopimo navigacijske simbole, ker so neuporabni
\setbeamertemplate{navigation symbols}{}
% Aktiviramo oštevilčenje strani
\setbeamertemplate{footline}[frame number]{}



% Naloga 1.3.3: definirajte okolji `definicija' in `izrek'.
% Namig: z iskanjem po datotekah (Ctrl+Shift+F oz. Cmd+Shift+F) 
% poiščite niz `{definicija}' ali niz `{izrek}'.
{\theoremstyle{definition}
\newtheorem{definicija}{Definicija}
}
{\theoremstyle{plain}
\newtheorem{izrek}{Izrek}
}

\begin{document}

\title{Matematični izrazi in uporaba paketa \texttt{beamer}}
\subtitle{\emph{Matematičnih} nalog ni treba reševati!}
\institute{Fakulteta za matematiko in fiziko}
\date{}
% Naloga 1.3.4: pripravite naslovno stran z vsebino:
% - naslov: Matematični izrazi in uporaba paketa \texttt{beamer}
% - podnaslov: \emph{Matematičnih} nalog ni treba reševati!
% - inštitut: Fakulteta za matematiko in fiziko
% - datum: naj se ne izpiše; to dosežete z ukazom \date{}.
% Zgornje podatke nastavite z ukazi kot v dokumentih razreda `article`.
% Več o tem, kako se naredi naslovno stran, si preberite na naslovu na naslovu:
% https://www.overleaf.com/learn/latex/Beamer
% To stran preberite do vključno razdelka "Creating a table of contents".
% Ukaz `\titlepage` deluje podobno kot ukaz `\maketitle`, ki ste ga že srečali.

\begin{frame}
    \titlepage
\end{frame}

% Naloga 1.3.5: pripravite kazalo vsebine.
% 1. Naslov prosojnice, s kazalom vsebine naj bo "Kratek pregled"
% 2. S pomožnim parametrom `pausesections' (v oglatih oklepajih) 
%    določite, da naj se kazalo vsebine odkriva postopoma.
%    Poglejte, kako deluje ta ukaz.
% 3. Ker ni videti preveč lepo, pomožni parameter zakomentirajte.

\begin{frame}
    \frametitle{Kratek pregled}
    \tableofcontents
\end{frame}



%\begin{frame}
%    \frametitle{Paket \texttt{beamer}}
%\end{frame}
\section{Paket \texttt{beamer}}
\input{prosojnice/1-paket-beamer.tex}
\begin{frame}
    
\end{frame}


\begin{frame}
    \section{Paketa \texttt{amsmath} in \texttt{amsfonts}}
\end{frame}


\begin{frame}
    \section[Matematika, 1. del\\\large{Analiza, logika, množice}]{Matematika, 1. del}
\end{frame}



\begin{frame}
    \section{Stolpci in slike}
\end{frame}


\begin{frame}
    \section{Paket \texttt{beamer} in tabele}
\end{frame}


\begin{frame}
    \section[Matematika, 2. del\\\large{Zaporedja, algebra, grupe}]{Matematika, 2. del}
\end{frame}

\end{document}